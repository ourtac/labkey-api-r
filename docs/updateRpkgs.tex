\documentclass[12pt]{article}

\title{Guide to Updating R Packages}
\author{Valerie Obenchain}	
\date{December 2008}

\begin{document}
\maketitle			


\noindent This document is a rough outline of how to update an existing R package in UNIX or 
Windows. The primary reference for creating R packages is the Writing R Extensions manual located at
\emph{http://cran.r-project.org/doc/manuals/R-exts.pdf}. An additional reference that may be helpful for building packages in Windows is\\
 \emph{http://faculty.chicagogsb.edu/peter.rossi/research/bayes book/bayesm/Making R Packages Under Windows.pdf}.

\section{UNIX}

To create an updated version tar.gz file for an existing R package, follow these general steps:
\begin{itemize}
\item {Make the desired change to the source code: R package structure is described in Section 1 of Writing R Extensions. The R source code is in the R subdirectory and if there is executable code it is in the exec subdirectory.}

\item {Check and build the package: Once the package is assembled, the check and build can be done from
the command line. \\
R CMD check $<$packagename$>$\\
R CMD build $<$packagename$>$}
\item {Once the check and build processes are error free or only have errors/warnings you are willing to allow, upload the package to CRAN. Detailed
instructions for upload are included in the Writing R Extensions manual mentioned above.}
\end{itemize} 



\section{Windows}
To create an updated version .zip file for an existing R package on a Windows machine, follow these general steps:

\begin{itemize}
\item{Install necessary tools: To build an R package in Windows you need to install some additional software tools. Download and install Rtools, Microsoft's HTML Help Workshop, MikTeX and InnoSetup installer which can be found at \emph {http://www.murdoch-sutherland.com/Rtools/}. }
\item{Change the PATH environment variable:
Add the directories containing the newly installed tools to the beginning of the path system variable.}
\item {Make the desired change to the source code: R package structure is described in Section 1 of Writing R Extensions. The R source code is in the R subdirectory and if there is executable code it is in the exec subdirectory. Note that the package directory used to create a .zip file on Windows is the same original package directory used to create the tar.gz on UNIX.}
\item{Check and build the package: This can be done from the command line as follows:\\
R CMD check $<$packagename$>$\\
R CMD build -binary $<$packagename$>$}
\item {Once the check and build processes are complete the .zip package is ready for installation on a Windows machine. Only the tar.gz file needs to be uploaded to CRAN. They do not accept binary submissions and they will make the .zip file for the CRAN site. Detailed
instructions for upload are included in the Writing R Extensions manual mentioned above.}

\end{itemize}



\end{document}            

