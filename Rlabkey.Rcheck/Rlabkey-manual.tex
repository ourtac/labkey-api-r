\documentclass{article}
\usepackage[ae,hyper]{Rd}
\begin{document}
\HeaderA{Rlabkey-package}{Import data from a labkey data base into an R data frame}{Rlabkey.Rdash.package}
\aliasA{Rlabkey}{Rlabkey-package}{Rlabkey}
\keyword{package}{Rlabkey-package}
\begin{Description}\relax
This package allows the import of data from a labkey data base
into an R data frame. The import can be accomplished with Sql
commands or by specifying the query schema information.
\end{Description}
\begin{Details}\relax
\Tabular{ll}{
Package: & Rlabkey\\
Type: & Package\\
Version: & 0.0.1\\
Date: & 2008-08-18\\
License: & Apache 2.0\\
LazyLoad: & yes\\
}
Need to insert text on how to use the package in general.
\end{Details}
\begin{Author}\relax
Valerie Obenchain
\end{Author}
\begin{References}\relax
http://www.omegahat.org/RCurl/,
http://dssm.unipa.it/CRAN/web/packages/rjson/rjson.pdf,
https://www.labkey.org/project/home/begin.view
\end{References}
\begin{SeeAlso}\relax
\code{\LinkA{labkey.selectRows}{labkey.selectRows}}, \code{\LinkA{labkey.executeSql}{labkey.executeSql}}, \code{\LinkA{makeFilter}{makeFilter}}
\end{SeeAlso}

\HeaderA{labkey.executeSql}{Retrieve data from a labkey database using Sql commands}{labkey.executeSql}
\keyword{IO}{labkey.executeSql}
\begin{Description}\relax
Use Sql commands to specify data to be imported into R. Prior to import, data can
be manipulated through all standard Sql commands.
\end{Description}
\begin{Usage}
\begin{verbatim}
labkey.executeSql(baseUrl, folderPath, schemaName, sql, maxRows = NULL, 
                                  rowOffset = NULL, stripAllHidden = TRUE)
\end{verbatim}
\end{Usage}
\begin{Arguments}
\begin{ldescription}
\item[\code{baseUrl}] a string specifying the \code{baseUrl}for the HTTP request
\item[\code{folderPath}] a string specifying the \code{folderPath} for the HTTP request
\item[\code{schemaName}] a string specifying the  \code{schemaName} for the HTTP request
\item[\code{sql}] a string containing the \code{sql} commands to be executed
\item[\code{maxRows}] (optional) an integer specifying how many rows of data to return. If no value is specified, all rows are returned.
\item[\code{rowOffset}] (optional) an integer specifying which row of data should be the first row in the retrieval. 
If no value is specified, all rows are returned.
\item[\code{stripAllHidden}] (optional) a logical value indicating whether or not to save data columns that would 
normally be hidden from user veiw. If no value is specified, no hidden columns are returned.
\end{ldescription}
\end{Arguments}
\begin{Details}\relax
A full dataset or user saved view can be imported into an R data frame using the \code{labkey.executeSql}
function. The function accepts as its arguments components of the url that identify the location of the
data and what Sql actions should be taken on the data prior to import. Data are returned in a data frame
with column names as they appear in on the labkey database website.
\end{Details}
\begin{Value}
The requested data are returned in a data frame.
\end{Value}
\begin{Author}\relax
Valerie Obenchain
\end{Author}
\begin{References}\relax
http://www.omegahat.org/RCurl/, 
http://dssm.unipa.it/CRAN/web/packages/rjson/rjson.pdf,
https://www.labkey.org/project/home/begin.view
\end{References}
\begin{SeeAlso}\relax
\code{\LinkA{labkey.selectRows}{labkey.selectRows}}
\end{SeeAlso}
\begin{Examples}
\begin{ExampleCode}

library(Rlabkey)

# Retrieve Participant ID, age and height from Demographics table
# on www.labkey.org
### NOTE: This won't work until 8.3 is up on www.labkey.org ####

#getdata <- labkey.executeSql(baseUrl="https://www.labkey.org", folderPath="/home/Study/demo",schemaName="study", sql="select Demographics.ParticipantId, Demographics.Age, Demographics.Height from Demographics")

\end{ExampleCode}
\end{Examples}

\HeaderA{makeFilter}{Builds an array of filters}{makeFilter}
\keyword{file}{makeFilter}
\begin{Description}\relax
This function takes inputs of column name, filter value and filter operator for
the data to be filtered on. It returns an array of filters to be used in \code{labkey.selectRows}
\end{Description}
\begin{Usage}
\begin{verbatim}
makeFilter(c("colname", "operator",value))
\end{verbatim}
\end{Usage}
\begin{Arguments}
\begin{ldescription}
\item[\code{colname}] a string specifying the name of the column to be filtered
\item[\code{operator}] a text string specifying what operator should be used in the filter
\item[\code{value}] an integer or string specifying the value the columns should be filtered on
\end{ldescription}
\end{Arguments}
\begin{Details}\relax
Possible operator values are as follows:
"EQUALS", "NOT\_EQUALS", "GREATER\_THAN", "GREATER\_THAN\_OR\_EQUAL\_TO", "LESS\_THAN",
"LESS\_THAN\_OR\_EQUAL\_TO", "DATE\_EQUAL", "DATE\_NOT\_EQUAL", "NOT\_EQUAL\_OR\_NULL",
"IS\_NULL", "IS\_NOT\_NULL", "CONTAINS", and "DOES\_NOT\_CONTAIN".

Multiple filters can be applied (see examples). Currently this function supports
specifying up to five filters.
\end{Details}
\begin{Value}
The function returns either a single string or an array of strings to be use in the
\code{colFilter} argument of the \code{labkey.selectRows} function.
\end{Value}
\begin{Author}\relax
Valerie Obenchain
\end{Author}
\begin{References}\relax
http://www.omegahat.org/RCurl/, 
http://dssm.unipa.it/CRAN/web/packages/rjson/rjson.pdf,
https://www.labkey.org/project/home/begin.view
\end{References}
\begin{SeeAlso}\relax
\code{\LinkA{labkey.selectRows}{labkey.selectRows}}
\end{SeeAlso}
\begin{Examples}
\begin{ExampleCode}

# Create filters
myfilters<- makeFilter(c("HIVLoadQuant","GREATER_THAN",500), c("HIVRapidTest","EQUALS","Positive"))

# Use in labkey.selectRows function
getdata <- labkey.selectRows(baseUrl="https://www.labkey.org", folderPath="/home/Study/demo", schemaName="study", queryName="HIV Test Results", colSelect=c("ParticipantId","HIVDate","HIVLoadQuant","HIVRapidTest"), colFilter=myfilters)


\end{ExampleCode}
\end{Examples}

\HeaderA{labkey.selectRows}{Retrieve data from a labkey database using url specifications}{labkey.selectRows}
\keyword{IO}{labkey.selectRows}
\begin{Description}\relax
Use url to specify data to be imported into R. Prior to import, data columns
can be sorted, specific columns or number of rows can be requested and
data filters can be applied.
\end{Description}
\begin{Usage}
\begin{verbatim}
labkey.selectRows(baseUrl, folderPath, schemaName, queryName, viewName = NULL, 
                                  colSelect = NULL, maxRows = NULL, rowOffset = NULL, colSort = NULL, 
                                  colFilter = NULL, stripAllHidden = TRUE)
\end{verbatim}
\end{Usage}
\begin{Arguments}
\begin{ldescription}
\item[\code{baseUrl}] a string specifying the \code{baseUrl}for the HTTP request
\item[\code{folderPath}] a string specifying the \code{folderPath} for the HTTP request
\item[\code{schemaName}] a string specifying the  \code{schemaName} for the HTTP request
\item[\code{queryName}] a string specifying the \code{queryName} for the HTTP request
\item[\code{viewName}] (optional) a string specifying the \code{viewName} for the HTTP request
\item[\code{colSelect}] (optional) a vector of comma separated strings specifying which columns of a dataset or view to import
\item[\code{maxRows}] (optional) an integer specifying how many rows of data to return. If no value is specified, all rows are returned.
\item[\code{colSort}] (optional) a string including the name of the column to sort preceeded by a "+" or "-" to indicate sort direction
\item[\code{rowOffset}] (optional) an integer specifying which row of data should be the first row in the retrieval. If no
value is specified, the retrieval starts with the first row.
\item[\code{colFilter}] (optional) a vector or array object created by the \code{makeFilter} function which contains the
column name, operator and value of the filter(s) to be applied to the retrieved data.
\item[\code{stripAllHidden}] (optional) a logical value indicating whether or not to save data columns that would normally be hiddenfrom user view. If no value is specified, no hidden columns are returned.
\end{ldescription}
\end{Arguments}
\begin{Details}\relax
A full dataset or user saved view can be imported into an R data frame using the \code{labkey.selectRows} 
function. The function accepts as its arguments the components of the url that identify
the location of the data and what actions should be taken on the data prior to import
(ie, sorting, selecting particular columns or maximum number of rows, etc.) Data are returned in a data 
frame with column names as they appear on the labkey database website. 

Use care when specifying column names for the colSelect or colFilter arguments. Often the column name
is not the same as the column header as seen on the web site. ***More help here*******

When importing data from ATLAS.scharp.org, a quick and simple way to identify the necessary components of the url 
(ie, schemaName, queryName, viewName, etc.) is to use the "export to R script" option avaiable as a drop down
under the "views" tab for each dataset.
\end{Details}
\begin{Value}
The requested data are returned in a data frame.
\end{Value}
\begin{Author}\relax
Valerie Obenchain
\end{Author}
\begin{References}\relax
http://www.omegahat.org/RCurl/, 
http://dssm.unipa.it/CRAN/web/packages/rjson/rjson.pdf,
https://www.labkey.org/project/home/begin.view
\end{References}
\begin{SeeAlso}\relax
\code{\LinkA{labkey.executeSql}{labkey.executeSql}}, \code{\LinkA{makeFilter}{makeFilter}}
\end{SeeAlso}
\begin{Examples}
\begin{ExampleCode}

## Retrieving data from the Labkey.org web site:

library(Rlabkey)

# Retrieve HIV Test Results and plot Western Blot data
getdata <- labkey.selectRows(baseUrl="https://www.labkey.org", folderPath="/home/Study/demo", 
                                schemaName="study", queryName="HIV Test Results")
plot(factor(getdata$"HIV Western Blot"), main="HIV Western Blot")

# Select columns and apply filters
myfilters<- makeFilter(c("HIVLoadQuant","GREATER_THAN",500), c("HIVRapidTest","EQUALS","Positive"))
getdata <- labkey.selectRows(baseUrl="https://www.labkey.org", folderPath="/home/Study/demo", schemaName="study", queryName="HIV Test Results", colSelect=c("ParticipantId","HIVDate","HIVLoadQuant","HIVRapidTest"), colFilter=myfilters)


\end{ExampleCode}
\end{Examples}

\end{document}
