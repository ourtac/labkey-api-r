\HeaderA{Rlabkey-package}{Import/export data between a labkey database and R}{Rlabkey.Rdash.package}
\aliasA{Rlabkey}{Rlabkey-package}{Rlabkey}
\keyword{package}{Rlabkey-package}
\begin{Description}\relax
This package allows the transfer of data between a labkey database and an R session. Data can be imported from 
a labkey database into R by specifying the query schema information (\code{labkey.selectRows}) 
or by using sql commands (\code{labkey.executeSql}). From an R session, existing data can be updated
(\code{labkey.updateRows}), new data can be inserted (\code{labkey.insertRows}) or  
data can be deleted from the labkey database (\code{labkey.deleteRows}). 

The user must have the appropriate authorization on the labkey
server in order to modify the database through the use of
these functions.
\end{Description}
\begin{Details}\relax
\Tabular{ll}{
Package: & Rlabkey\\
Type: & Package\\
Version: & 0.0.3\\
Date: & 2008-09-02\\
License: & Apache 2.0\\
LazyLoad: & yes\\
}
Using this package to access a password protected labkey data base requires that the user
has their login information in a netrc file. The netrc file
contains configuration and autologin information for the File Transfer Protocol client (ftp) and
other programs such as CURL.

On a UNIX system this file should be named .netrc (dot netrc) and on windows it sould be 
named \_netrc (underscore netrc). The file should be located in the users home directory and the 
permissions on the file should be unreadable for everybody except the owner.  

To create the \_netrc on a windows machine, first create an environment variable called 'HOME' that 
is set to your home directory (c:/Users/<User-Name> on Vista) or any directory you want to use. 
In that directory, create a text file named \_netrc (note that it's underscore netrc, not dot 
netrc like it is on UNIX). 

The following three lines must be included in the .netrc or \_netrc file either separated by white space
(spaces, tabs, or newlines) or commas.

machine <remote-machine-name>\\
login <user-email>\\
password <user-password>


One example would be:\\
machine atlas.scharp.org\\
login vobencha@fhcrc.org\\
password mypassword\\

Another example would be:\\
machine atlas.scharp.org login vobencha@fhcrc.org password mypassword
\end{Details}
\begin{Author}\relax
Valerie Obenchain
\end{Author}
\begin{References}\relax
http://www.omegahat.org/RCurl/,\\
http://dssm.unipa.it/CRAN/web/packages/rjson/rjson.pdf,\\
https://www.labkey.org/project/home/begin.view
\end{References}
\begin{SeeAlso}\relax
\code{\LinkA{labkey.selectRows}{labkey.selectRows}}, \code{\LinkA{labkey.executeSql}{labkey.executeSql}}, \code{\LinkA{makeFilter}{makeFilter}}, 
\code{\LinkA{labkey.insertRows}{labkey.insertRows}}, \code{\LinkA{labkey.updateRows}{labkey.updateRows}}, \code{\LinkA{labkey.deleteRows}{labkey.deleteRows}}
\end{SeeAlso}

